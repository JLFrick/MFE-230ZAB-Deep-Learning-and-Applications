\documentclass[11pt]{article}
\usepackage{hyperref} 
\usepackage{amsmath, amsfonts, amssymb}
\usepackage{graphicx}
\usepackage{float}
\usepackage[backend=biber,style=ieee,sorting=ynt]{biblatex}
\addbibresource{references.bib} 
\usepackage[margin=1in]{geometry}

\emergencystretch=0pt
\pretolerance=150
\tolerance=10000
\hbadness=10000
\hfuzz=0pt

\title{230ZB Notes}
% \author{}
\date{\today} 

\begin{document}
\maketitle 
% \pagebreak
% \tableofcontents 
% \pagebreak

% \section{Introduction}

So overall idea I was thinking was to have GPT forecast which GICS \{sector|industry group|industry\} 
will have the highest returns over the next day (not tied to the idea in case anyone else has 
something they'd rather do). Currently, the model is running at the industry group level (which has 
more categories than the sector level but far less than the industry level). I was thinking maybe more 
granularity may be better, since it's such a short term strategy. 


All three files should run top to bottom right now. File 1 \verb|1_get_data.ipynb| basically just deals with 
getting the raw data. It does the following:
\begin{enumerate}
    \item Get S\&P constituent data (GICS membership, ticker, returns, date, etc.) and is has updated 
    holdings as the constituents come in and out of the S\&P.
    \item Gets all the FOMC statements, scrapes them, and removes noise.
\end{enumerate}


Then the next file, \verb|2_construct_fine_tune_dataset.ipynb| constructs the datasets needed 
to finetune the model. If the FOMC statements and their date is the data, the correct label 
given our objective is the GICS \{sector|industry group|industry\} that has the highest and lowest
returns over the next day. This file gets those correct labels, structures the data in the expected 
format for the fine tuning process, and then ensures the data is in the proper format.


The last file I pushed is the \verb|3_gics_strategy.ipynb|. This one basically fine tunes the 
model, runs the model, and constructs the porfolio. I'd say this is kind of 
just a rough draft version that still needs some work. Also, the instructions 
for this is quite vague.



\end{document}
